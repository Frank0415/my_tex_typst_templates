\documentclass{beamer}

\usepackage{focs-slides}

\title{Presentation Topic}
\author{Team member names}
\date{25SP}
\course{ece4821}

% bibliography file
\addbibresource{project.bib}

% \definecolor{darkblue}{HTML}{6666dd} 
% \definecolor{darkslategray}{HTML}{2f4f4f} 
% \definecolor{ebpflightgreen}{HTML}{a3c9c3}
\definecolor{rustorange}{HTML}{df7a4c}
\definecolor{rustdarkblue}{HTML}{4e4e73}
\definecolor{rustlightblue}{HTML}{69839d}
\definecolor{rustspgreen}{HTML}{61784D}
%\colortheme{green!42!black}
%\colortheme{orange!85!black}
%\colortheme{darkblue}
% \colortheme{pink!80!black}
%\colortheme{orange!85!white!90!black}
% \colortheme{darkslategray}
\colortheme{rustorange}

\begin{document}

\maketitle

% table of contents
%\toc{enum}
%\toc{mindmap}

\section{Section}

% \begin{frame}{Frame Title}
%     % Insert image or graphic here
%     \includegraphics[width=\columnwidth]{repeat2}
% \end{frame}

\begin{frame}{Frame Title}
    \em Placeholder for discussion or question. \em
    \bigskip\raggedright
    \pause
    \begin{itemize}
        \item First bullet point
        \item Second bullet point
    \end{itemize}
\end{frame}

\begin{frame}{Frame Title}
    Solution: Use \textbf{design patterns}!
    \begin{definition}
        \textbf{Definition placeholder.}
    \end{definition}
\end{frame}

\section{Section}

\begin{frame}[fragile]{Frame Title}
    Placeholder for code example:
    \begin{codetbox}{elm}
-- Elm code example
    \end{codetbox}
\end{frame}

\begin{frame}[fragile]{Frame Title}
    Placeholder for code call function:
    \begin{codetbox}{elm}
-- Elm code example
    \end{codetbox}
    % Add commentary or questions here
    \pause
    \em Placeholder for follow-up question. \em
\end{frame}

\begin{frame}[fragile]{Frame Title}
    Placeholder for operator or function explanation:
    \begin{codetbox}{elm}
-- Elm code example
    \end{codetbox}
    \begin{codetbox}{elm}
-- Elm code example
    \end{codetbox}
\end{frame}

\begin{frame}[fragile]{Frame Title}
    Placeholder for more calculations:
    \pause
    \begin{codetbox}{elm}
-- Elm code example
    \end{codetbox}
\end{frame}

\begin{frame}[fragile]{Frame Title}
    Placeholder for alternative operator:
    \begin{codetbox}{elm}
-- Elm code example
    \end{codetbox}
    \pause
    \begin{codetbox}{elm}
-- Elm code example
    \end{codetbox}
    \pause
    \em Placeholder for data type question. \em
\end{frame}

\section{Section}

\begin{frame}[fragile]{Frame Title}
    Placeholder for calculations:
    \begin{codetbox}{elm}
-- Elm code example
    \end{codetbox}
    \pause
    \em Placeholder for tracing question. \em
\end{frame}

\begin{frame}[fragile]{Frame Title}
    \begin{codetbox}{elm}
-- Elm code example
    \end{codetbox}
    \pause
    Issue 1: Placeholder for issue description.
\end{frame}

\begin{frame}[fragile]{Frame Title}
    Placeholder for wrapping function:
    \begin{codetbox}{elm}
-- Elm code example
    \end{codetbox}
    New call function:
    \begin{codetbox}{elm}
-- Elm code example
    \end{codetbox}
    \pause
    Issue 2: Placeholder for issue description.
\end{frame}

\begin{frame}[fragile]{Frame Title}
    Placeholder for logic extraction:
    \begin{codetbox}{elm}
-- Elm code example
    \end{codetbox}
    Here transform can be fed with calculation functions.
\end{frame}

\begin{frame}[fragile]{Frame Title}
    Change the remaining part of the code accordingly:
    \begin{codetbox}{elm}
-- Elm code example
    \end{codetbox}
\end{frame}

\begin{frame}[fragile]{Frame Title}
    Change call function for better readability:
    \begin{codetbox}{elm}
-- Elm code example
    \end{codetbox}
    \pause
    This is called the \textit{Monadic} style.
    \pause
    Placeholder for explanation of differences and extensibility.
\end{frame}

\begin{frame}{Exercise}
    \begin{itemize}
        \item Add new functions and modify call functions to test them.
        \item Find the similarities between the examples.
    \end{itemize}
\end{frame}

\begin{frame}{Frame Title}
    Placeholder for similarities and explanation.
    \medskip\pause
    Placeholder for further explanation.
    \medskip\pause
    \em Placeholder for quote. \em
\end{frame}

\section{Section}

\begin{frame}{Frame Title}
    A monad contains three things:
    \begin{itemize}
        \item \textbf{A type constructor} placeholder.
        \item \textbf{A function} placeholder.
        \item \textbf{A function} placeholder.
    \end{itemize}
    \medskip\pause

    \em Identify the three components in the examples above. \em
    \medskip\pause

    \em Placeholder for quote. \em
\end{frame}

\begin{frame}[fragile]{Frame Title}
    Placeholder for monad exercise and explanation.

    \medskip\pause

    \textbf{Exercise:}
    List is also a Monad! Try to find the constructor, return and bind functions for it. Then, write a function that returns a list of all possible results of multiplying two integers in two respective lists.

    \textbf{Hint}: Placeholder for hint.
\end{frame}

\begin{frame}[fragile]{Frame Title}
    Placeholder for package installation and usage.
    \begin{codetbox}{bash}
elm install package/name
    \end{codetbox}
    \begin{codetbox}{elm}
import Module.Name exposing (..)
    \end{codetbox}
    \medskip\pause
    \em Rewrite the previous exercise with the package. \em
\end{frame}

\begin{frame}{Frame Title}
    \begin{itemize}
        \item Placeholder for general design pattern explanation.
        \item Placeholder for interface/abstract class analogy.
        \item Placeholder for multiple monads for one container type.
        \item Placeholder for helper functions.
    \end{itemize}
\end{frame}

% \begin{framek}[Conclusion]
%     Placeholder for conclusion and further resources.
%     \medskip
%     \begin{itemize}
%         \item Youtube. \href{https://www.youtube.com/}{Video Title}
%         \item Wiki. \href{https://wiki.example.com/}{Wiki Title}
%         \item Wikipedia. \href{https://en.wikipedia.org/}{Wikipedia Title}
%     \end{itemize}
% \end{framek}

\thankframe

\end{document}
